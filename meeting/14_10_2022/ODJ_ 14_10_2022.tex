\documentclass[a4paper]{article}
\usepackage[boxedEnv, titleFormat]{myBelovedPackage}

\margin{3cm}{4cm}


% Vraisemblance
\newcommand{\Lmarg}[1][]{\mc L_{marg#1}}
\newcommand{\logmarg}[1][]{\ell_{marg#1}}

\newcommand{\Lcomp}[1][]{\mc L_{comp#1}}
\newcommand{\logcomp}[1][]{\ell_{comp#1}}

\newcommand{\varobs}{\mc O}
\newcommand{\varlat}[1][]{Z#1}
\newcommand{\pen}{\mathrm{pen}}
\DeclareMathOperator*{\prox}{Prox}


\newcommand{\hazard }{h}
\newcommand{\regularization }{\lambda}
\newcommand{\nonlinearfct}{m}


\newcommand{\forany}[3][1]{#1\leq #2 \leq #3}
%Indice
\newcommand{\geno}{g}
\newcommand{\maxgeno}{G}
\newcommand{\foranygeno}{\forany{\geno}{\maxgeno}}

\newcommand{\lines}{\ell}
\newcommand{\maxlines}{L}
\newcommand{\foranylines}{\forany{\lines}{\maxlines}}

\newcommand{\genolines}{ {\geno,\lines} }
\newcommand{\meltgenolines}{ {\geno\lines} }

\newcommand{\obs}{j}
\newcommand{\maxobs}{J}
\newcommand{\foranyobs}{\forany{\obs}{\maxobs}}

\newcommand{\genolinesobs}{ {\geno, \lines, \obs} }
\newcommand{\genoobs}{ {\geno, \obs} }




% --- lazy math command --- %
\newcommand{\psinorm}[2]{\frac 12 \log(2\pi#1^2) + \frac{#2^2}{2#1^2}}
\newcommand{\psibar}[1]{\psinorm{\bar #1}{\sigma_{#1}}}



% Petite commande pour faire des model exp = h + <S, Phi> + psi parce que je suis feignant
\newcounter{Sc}
\setcounter{Sc}{0}
\newcommand{\Scount}{{\stepcounter{Sc}\theSc}}

\newcounter{Phic}
\setcounter{Phic}{0}
\newcommand{\Phicount}{{\stepcounter{Phic}\thePhic}}

\newcounter{hc}
\setcounter{hc}{0}
\newcommand{\hcount}{{\stepcounter{hc}\thehc}}

\newcounter{psic}
\setcounter{psic}{0}
\newcommand{\psicount}{{\stepcounter{psic}\thepsic}}
    

% === Couleur === %
\setmyColor[ blue = {0, 0, 255},
             violet = {163, 73, 164},
             fg = {0, 163, 166}]

\xdefinecolor{inraeRed}{RGB}{237, 110, 108}
\xdefinecolor{inraeBlue}{RGB}{66, 48, 137}
\xdefinecolor{inraeCyan}{RGB}{0, 163, 166}
\xdefinecolor{inraeGreen}{RGB}{157, 197, 68}

\newcommand{\inrae}[1]{\tc{inraeCyan}{\bf{#1}}}
\newcommand{\known}[1]{ \tc{BrickRed}{#1}}

\newcommand{\by}[1]{\textbf{[#1]}}


% === enumerate === %


\newenvironment{myEnumerate}[2][white]{
    \setlist[enumerate]{%wide=0pt,
                        %labelsep=0.5em,
                        font=\large\color{#1}}
    
    \setlist[enumerate,1]{label = \colorbox{#2}{\makebox[0.75em][c]{\arabic*}}}
    \setlist[enumerate,2]{label = \colorbox{#2}{\makebox[0.75em][c]{\alph*}}}
    \setlist[enumerate,3]{label = \colorbox{#2}{\makebox[0.75em][c]{\roman*}}}

    \begin{enumerate}
    }{ \end{enumerate} }



\newcommand{\remark}[1]{\tc{inraeRed}{#1}}
\newenvironment{myRemarks}{
    \color{inraeRed}
    \begin{myEnumerate}{inraeRed} 
    }{ \end{myEnumerate} }
    

\begin{document} 
\begin{myText}\centering

{\Large Ordre du jour : \underline{Réunion de rentrée de thèse}\\ 14 octobre 2022}

\vspace{2cm}

\begin{myEnumerate}{inraeCyan}
  \item Rappel du modèle du stage (notation, etc) : \by{Antoine}

  \item Conclusion du stage : \by{Antoine}
      \begin{myEnumerate}{inraeBlue}
        \item ce qui a été fait,
      \end{myEnumerate}

  \item Perspective du stage (à finir ?) : \by{Antoine/Discutions}
      \begin{myEnumerate}{inraeBlue}
        \item Réussir à inférer tous les paramètres du modèle,
        \item Tester d'autre pénalisation, \remark{pour quelle donnée ?}
        \item Application donnée réelle ?
            \\ \remark{Idée de tristan : kinship / matrice de ressemblance}
      \end{myEnumerate}
  
  \item  Discussion et sujet à aborder :
      \begin{myEnumerate}{inraeBlue}
        \item Mes questions/précisions à apporter sur la bio, \by{Antoine}
        
        \begin{myRemarks}
            \item Fond génétique : 
            \\ c'est quoi la différence génétique entre deux lignées dans un même fond
            \\ item Qu'est-ce qui fait que les dates différent ?
    
            \item J'ai pas compris les mécanismes de défense du mais.
            \\En particulier est-ce bien de fleurir lors d'une attaque ?

            \item Comment savez vous que vous avez des pyrales bivoltines et non deux pyrales différent ? voire les bivoltines et les univoltines ?
        \end{myRemarks}
    
        
        \item apparition de la pyrale dans le champ \by{Discutions}            
            \begin{myRemarks}
                \item Thèse de Brigitte got encadrée françois rodolphe
                Approche de la modélisation de la dynamique de population de la pyraled de maïs dans le bassin parisien
                \item données plante à plante
            \end{myRemarks}
        \item propagation de la pyrale entre les plantes : \by{Julie/Elodie}
            \\ comment modéliser ? (Modélisation spatiale),
        \item impact du climat sur la pyrale, \by{Julie/Elodie}
            \\ ajout de covariable dans le model ? ou ?
            \begin{myRemarks}
                \item De quel climat à besoin de la pyrale ?
            \end{myRemarks}
      \end{myEnumerate}
  
  \item Avec Judit et Elodie :
      \\ comment gérer la double localisation Jouy/Moulon (administratif, ordre de mission,  etc)
      \begin{myRemarks}
          \item pass navigo
          \item télétravail 
          \item ordre de mission
      \end{myRemarks}
\end{myEnumerate}

\end{myText}
\end{document}