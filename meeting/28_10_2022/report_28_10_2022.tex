\documentclass[a4paper]{article}
\usepackage[boxedEnv, numbering, titleFormat, english]{myBelovedPackage}

\margin{3cm}{4cm}
\setlhead{Antoine Caillebotte}

\loadbiblatex[citestyle = authoryear]{references.bib}


% Vraisemblance
\newcommand{\Lmarg}[1][]{\mc L_{marg#1}}
\newcommand{\logmarg}[1][]{\ell_{marg#1}}

\newcommand{\Lcomp}[1][]{\mc L_{comp#1}}
\newcommand{\logcomp}[1][]{\ell_{comp#1}}

\newcommand{\varobs}{\mc O}
\newcommand{\varlat}[1][]{Z#1}
\newcommand{\pen}{\mathrm{pen}}
\DeclareMathOperator*{\prox}{Prox}


\newcommand{\hazard }{h}
\newcommand{\regularization }{\lambda}
\newcommand{\nonlinearfct}{m}


\newcommand{\forany}[3][1]{#1\leq #2 \leq #3}
%Indice
\newcommand{\geno}{g}
\newcommand{\maxgeno}{G}
\newcommand{\foranygeno}{\forany{\geno}{\maxgeno}}

\newcommand{\lines}{\ell}
\newcommand{\maxlines}{L}
\newcommand{\foranylines}{\forany{\lines}{\maxlines}}

\newcommand{\genolines}{ {\geno,\lines} }
\newcommand{\meltgenolines}{ {\geno\lines} }

\newcommand{\obs}{j}
\newcommand{\maxobs}{J}
\newcommand{\foranyobs}{\forany{\obs}{\maxobs}}

\newcommand{\genolinesobs}{ {\geno, \lines, \obs} }
\newcommand{\genoobs}{ {\geno, \obs} }




% --- lazy math command --- %
\newcommand{\psinorm}[2]{\frac 12 \log(2\pi#1^2) + \frac{#2^2}{2#1^2}}
\newcommand{\psibar}[1]{\psinorm{\bar #1}{\sigma_{#1}}}



% Petite commande pour faire des model exp = h + <S, Phi> + psi parce que je suis feignant
\newcounter{Sc}
\setcounter{Sc}{0}
\newcommand{\Scount}{{\stepcounter{Sc}\theSc}}

\newcounter{Phic}
\setcounter{Phic}{0}
\newcommand{\Phicount}{{\stepcounter{Phic}\thePhic}}

\newcounter{hc}
\setcounter{hc}{0}
\newcommand{\hcount}{{\stepcounter{hc}\thehc}}

\newcounter{psic}
\setcounter{psic}{0}
\newcommand{\psicount}{{\stepcounter{psic}\thepsic}}
    

\renewcommand{\varobs}{Y}
\renewcommand{\logmarg}{\ell}


\newenvironment{myEnumerate}[2][white]{
    \setlist[enumerate]{%wide=0pt,
                        %labelsep=0.5em,
                        font=\large\color{#1}}
    
    \setlist[enumerate,1]{label = \colorbox{#2}{\makebox[0.75em][c]{\arabic*}}}
    \setlist[enumerate,2]{label = \colorbox{#2}{\makebox[0.75em][c]{\alph*}}}
    \setlist[enumerate,3]{label = \colorbox{#2}{\makebox[0.75em][c]{\roman*}}}

    \begin{enumerate}
    }{ \end{enumerate} }

\begin{document} 
\begin{myText}

\begin{center}\Large{Compte rendu de travail pour le 28/10/2022} \end{center}

\section{Ce que j'ai fait cette semaine}

\begin{myEnumerate}{blue}
    \item En section \ref{complexity}, la formulation du calcul de $Q$ par récurrence ainsi qu'un petit paragraphe sur sa complexité.
    \item En section \ref{EM} et \ref{dempster} j'ai travaillé  sur \cite{dempster_maximum_1977} en particulier la démonstration de convergence,
    Je n'ai fait que reprendre, reformuler pour pouvoir l'assimiler. J'avais en particulier besoin de comprendre ce qu'était un GEM.
    \bf{Vous pouvez omettre cette partie}
    
    \item En section \ref{proxsection}, j'ai détaillé l'intégration du proximal et de la pénalisation en reprenant  \cite{fort_stochastic_2017}.
    
    \item En section \ref{modelsection}, j'ai voulu par curiosité voir à quoi ressemblerait le modèle joint inspiré par \cite{he_simultaneous_2015}.
    \tcr{J'ai également écrit une question de modélisation que je n'ai pas réussi à répondre seul !}
    \item En annexe \ref{SARGDM}, j'ai remis l'algorithme corrigé du stage que je vous avez envoyé il y 2 semaines, Si jamais vous voulez comparer avec l'algo de la section \ref{proxsection}. Mais ce sont pour moi les mêmes  !
\end{myEnumerate}

pvi : J'ai crée un répertoir git de partage : \href{https://github.com/caillebotteAntoine/Stat4Plant_WP3/}{Stat4Plant\_WP3}


\newpage

\section{SAEM without exponential family}\label{complexity}

\bf{We wish to write here the approximation step in a general context according to the likelihood only.}

Assuming that the A step of the SAEM in the general context is written : 

$$Q_{k+1} (\theta) = (1-u_{k+1})Q_k (\theta) + u_{k+1} \logcomp(\theta; \varlat[^{(k+1)}];\varobs)$$

where $\logcomp = \log\Lcomp$, $\varlat$ is some latent variable and $\varobs$ the observation.


\begin{prop}
    $$Q_{k+1} (\theta) = \left(\pro l1{k+1} (1-u_l)\right)Q_0 (\theta) + \som l1{k+1} \left(\pro hlk (1-u_{h+1})\right) \times u_l \logcomp(\theta; \varlat[^{(l)}];\varobs)$$
\end{prop}

\begin{dem}
    Let us define the recurrence proposal according to our proposition. For $k=0$, we have obviously that : $Q_1 = (1-u_1)Q_0(\theta) + u_1 \logcomp(\theta; \varlat[^{(1)}];\varobs) $ so $\mc P_0$ true !

    Let be for a certain $k\geq 0$, suppose that $\mc P_k$ is true, we have :

    \begin{align*}
        Q_{k+2} 
        =& (1-u_{k+2}) Q_{k+1} + u_{k+2} \logcomp(\theta; \varlat[^{(k+2)}];\varobs)
        \\
        =& (1-u_{k+2}) \left( \pro l1{k+1} (1-u_l)\right)Q_0 (\theta)
         \\ & + \som l1{k+1} \left(\pro hlk (1-u_{h+1})\right) \times u_l \logcomp(\theta; \varlat[^{(l)}];\varobs)
         \\ & + u_{k+2}\logcomp(\theta; \varlat[^{(k+2)}];\varobs)  
        \\
        =& \left( \pro l1{k+2} (1-u_l)\right)Q_0 (\theta)
              + \som l1{k+1} \left(\pro hl{k+1} (1-u_{h+1})\right) \times u_l \logcomp(\theta; \varlat[^{(l)}];\varobs)
         \\ & + u_{k+2}\logcomp(\theta; \varlat[^{(k+2)}];\varobs)  
        \\
        =& \left( \pro l1{k+2} (1-u_l)\right)Q_0 (\theta)
            + \som l1{k+2} \left(\pro hl{k+1} (1-u_{h+1})\right) \times u_l \logcomp(\theta; \varlat[^{(l)}];\varobs)
    \end{align*}
    QED
\end{dem}

So if we initiate $Q$ with $Q_0 = 0$, we have $Q_{k+1}(\theta) = \som l1{k+1} \left(\pro hlk (1-u_{h+1})\right) \times u_l \logcomp(\theta; \varlat[^{(l)}];\varobs)$. Suppose we want to run a SAEM for a mixed effect model with $N \times J$ observations where $N$ is the number of individuals and $J$ is the number of observations for each individual. Suppose that the mixed effect variables are of dimension $N$ and that the maximization step is in a constant time $O(1)$. Let’s say that the computation of \bf{$\logcomp$ takes a constant time} and depend only on the total number of observations so computation will be in $O(NJ)$. Achieve the approximation step, i.e., the computation of $Q_k$,  will take about a linear time $O(k\times NJ)$. To compare, the calculation of the simulation step, following a one-step MCMC procedure,  is computed in $O(N\times J)$ if it has been well coded and in the more casual case, i.e., without optimization procedures, in $O(N\times NJ)$.

To conclude if we do a total of $K$ iteration of the SAEM, the complexity of all the approximation step will be $O(K^2 NJ)$. The simulation step is done in $O(KNJ$). \bf{This algorithm has a quadratic complexity in time depending on the number of iterations !}
















\newpage
\section{EM algorithm definition}\label{EM}

The goal of the algorithm is to find a local maximum likelihood of a model that depends on unobserved latent variables :
$$\argmax_{\theta\in\setr^p}\{\log\Lmarg(\varobs,\theta)\}$$

The EM focus on maximizing the quantity $Q(\theta|\theta_k)$ rather than directly improve $\log\Lmarg(Y;\theta)$ (\cite{dempster_maximum_1977}).

\begin{myAlgorithm}[10cm]
    \caption{Expectation Maximization}
    \Require{Number of iterations $K\geq 1$}
    \Initialize $\theta_0 \in \setr^d$ initial parameter value
    
    \For{$k=1$ \KwTo $K$}{
        \myBullet[red] \bf{Step E :} \\
         \quad computation of $Q(\theta |\theta_k) = \mbb E [ \log\Lcomp(\varobs, \varlat; \theta) | \varobs ; \theta_k]$
         \\
         \myBullet[red] \bf{Step M :} \\
         \quad  maximization of $Q( . |\theta_k)$ : $\theta_{k+1} = \argmax_{\theta\in\setr^p} Q(\theta |\theta_k)$
    }
     \Return{$\hat \theta = \theta_K$}
\end{myAlgorithm}

\section{General properties from \cite{dempster_maximum_1977}}\label{dempster}

We introduce the notation for the conditional density of $\varlat$ given $\varobs$ and $\theta$ : $p(\varlat|\varobs; \theta) = \frac{\Lcomp(\varobs, \varlat; \theta)}{\int \Lcomp(\varobs, \varlat; \theta) d\varlat}  = \frac{\Lcomp(\varobs, \varlat; \theta)}{\Lmarg(\varobs; \theta)}$. As $\log p(\varlat|\varobs; \theta) = \log \Lcomp(\varobs, \varlat; \theta) - \log \Lmarg(\varobs; \theta)$, the log marginal likelihood can be written : 

\begin{center}
    \fbox{$\log\Lmarg(Y;\theta) = \log \Lcomp(\varobs, \varlat; \theta) - \log p(\varlat|\varobs; \theta)$}
\end{center}

We take the expectation value of the observed data $\varlat$ given an estimate of the parameter $\theta'$. To do this we multiply by the density of $\varlat$ and integrate over $\varlat$ :

\begin{align*}    
\log\Lmarg(Y;\theta) =& \int\log \Lcomp(\varobs, \varlat; \theta)p(Z|Y,\theta')dz - \int \log p(\varlat|\varobs; \theta)p(Z|Y,\theta') dz
\\ =&
\mbb E[\log \Lcomp(\varobs, \varlat; \theta) | \varobs,\theta'] -
\mbb E[\log p(\varlat|\varobs; \theta) | \varobs,\theta']
\intertext{We use the already established notation : \fbox{$\stackeq[rcl]{ Q(\theta|\theta') &=& \mbb E[\log \Lcomp(\varobs, \varlat; \theta) | \varobs,\theta'] \\ H(\theta|\theta') &=& \mbb E[\log p(\varlat|\varobs; \theta) | \varobs,\theta']}$} }
\\ \log\Lmarg(Y;\theta) =& Q(\theta|\theta') - H(\theta|\theta')
\end{align*}

\tinypage[center]{8cm}{
\begin{lemme}[Lemma 1 from \cite{dempster_maximum_1977}]
    $$H(\theta', \theta) \leq H(\theta|\theta), \forall (\theta',\theta) \in \setr^p\times \setr^p$$
\end{lemme}}
    \afaire{Je n'ai pas réussi / approfondie la démo de ce lemme. Il se fait apparement avec des formules de Jensen  provenant d'un ouvrage que je n'ai pas trouvé en libre accès.}




\begin{defi}[Generalized-EM algorithm (GEM)]
    Let $M$ be a mapping : $\theta\mapsto M(\theta)$ from $\setr^p$ to $\setr^p$ such that each step $\theta_k\rightarrow\theta_{k+1}$ of the EM is defined by : 
    $$\theta_{k+1} = M(\theta)$$
    \\ An iterative algorithm with mapping $M$ is a \textbf{generalized-EM algorithm} if : 
    $$Q(M(\theta)|\theta) \geq Q(\theta|\theta), \forall \theta\in\setr^p$$
\end{defi}

\tinypage[center]{10cm}{
\begin{theo}[Theorem 1 of \cite{dempster_maximum_1977}]
    For every GEM algorithm, we have : 
    $$\log\Lmarg(\varobs; M(\theta)) \geq \log\Lmarg(\varobs;\theta), \forall \theta\in\setr^p$$
\end{theo}}







\section{Introduction of the proximal operator in the EM}\label{proxsection}

We want to introduce a penalty term in the likelihood and have the following optimization problem: 
\begin{center}
    \shabox{$\argmax_{\theta\in\setr^p} \{\log\Lmarg(\varobs,\theta) - \pen(\theta) \}$}
\end{center}

We want to use the EM algorithm but with the penalty term, as :\\
\begin{myAlgorithm}[10cm]
    \caption{Expectation Maximization - penalized}\label{EM-pen}
    \Require{Number of iterations $K\geq 1$}
    \Initialize $\theta_0 \in \setr^d$ initial parameter value
    
    \For{$k=1$ \KwTo $K$}{
        \myBullet[red] \bf{Step E :} \\
         \quad computation of $Q(\theta |\theta_k) = \mbb E [ \log\Lcomp(\varobs, \varlat; \theta) | \varobs ; \theta_k]$
         \\
         \myBullet[red] \bf{Step M :} \\
         \quad  maximization of $Q( . |\theta_k)$ : $\theta_{k+1} = \argmax_{\theta\in\setr^p} \{Q(\theta |\theta_k) - pen(\theta)\}$
    }
     \Return{$\hat \theta = \theta_K$}
\end{myAlgorithm}


\citeauthor{fort_stochastic_2017} proposes the following algorithm to obtain a maximum likelihood. \textbf{The presence of the proximal operator is explained in the following} (see \ref{prox_arrival}).

\begin{myAlgorithm}[12cm]
    \caption{proximal Expectation Maximization - penalized}\label{prox_EM}
    \Require{Number of iterations $K\geq 1$; a sequence of steps $\gamma_k > 0$}
    \Initialize $\theta_0 \in \setr^d$ initial parameter value
    
    \For{$k=1$ \KwTo $K$}{
        \myBullet[red] \bf{Step E :} \\
         \quad computation of $Q(\theta |\theta_k) = \mbb E [ \log\Lcomp(\varobs, \varlat; \theta) | \varobs ; \theta_k]$
         \\
         \myBullet[red] \bf{Step M :} \\
         \quad  proximal-gradient descent : $\theta_{k+1} = \prox_{\gamma, \pen} \{\theta_k + \gamma_{n+1} \nabla_1 Q(\theta |\theta_k) \}$
    }
     \Return{$\hat \theta = \theta_K$}
\end{myAlgorithm}

where $\nabla_1$ denotes the gradient operator according to the first variable, and the $\prox$ operator is defined as :

\tinypage[center]{10cm}{
\begin{defi}[Proximal operator]
    Let $\gamma$ be a positive step size,
    $$prox_{\gamma, \pen}(\theta) = \argmin_{\theta'\in\setr^p} \left(\pen(\theta') + \frac 1{2\gamma}\norm{\theta-\theta'}_2^2 \right)$$
\end{defi}}

\tinypage[center]{14cm}{
\begin{defi}[Generalized-EM-penalized algorithm (GEM-pen)]
    Let $M$ be a mapping : $\theta\mapsto M(\theta)$ from $\setr^p$ to $\setr^p$ such that each step $\theta_k\rightarrow\theta_{k+1}$ of the EM is defined by : 
    $$\theta_{k+1} = M(\theta)$$
    \\ An iterative algorithm with mapping $M$ is a \textbf{generalized-EM-pen algorithm} if : 
    $$Q(M(\theta)|\theta)-\pen(M(\theta)) \geq Q(\theta|\theta)-\pen(\theta), \forall \theta\in\setr^p$$
\end{defi}}

\afaire{Je ne sais pas si cette définition est original au papier d'Ollier}


\begin{prop}[Proposition 1 from \cite{fort_stochastic_2017}]
    Let assume that $\pen : \setr^p \rightarrow \setr$ is convex and lower semi-continuous,  
    \\and that there exists a constant $L>0$ such that for any $\theta'\in\setr^p$ the gradient of $\theta \mapsto Q(\theta, \theta')$ is $L$-Lipschitz.
    \\Let $(\gamma_k)_{k\geq 0}$  be a positive sequence of step-size, \textbf{such that $\gamma_k\in ]0,1/L ], \forall k\geq 0$}.
    \\Then the \autoref{prox_EM} is a GEM-algorithm for the maximization of $\log\Lmarg - \pen$
\end{prop}

\newpage 
\begin{dem}
    \tinypage[center]{12cm}{
    \begin{lemme}
        Under the assumptions of \ref{prox_arrival}, for any $\gamma \in ]0,1/L]$ and $\theta, \theta'\in\setr^p\times \setr^p$
        $$Q(\theta|\theta') \geq Q(\theta|\theta')
                                    -\frac 1{2\gamma} \norm{\gamma \nabla_1 Q(\theta', \theta') +\theta' - \theta}^2
                                    +\frac \gamma 2 \norm{\nabla_1 Q(\theta', \theta')}^2$$

        where $\nabla_1$ denotes the gradient operator according to the first variable
    \end{lemme}}
    \begin{dem}
    Let $\theta, \theta'\in\setr^p\times \setr^p$,
    
    Recall for any $\theta'\in\setr^p$ the gradient of $Q(., \theta')$ is $L$-Lipschitz, so the Taylor's theorem give to ordre 1 at $\theta'$

    \begin{align*}
        Q(\theta|\theta') 
           & \geq Q(\theta'|\theta') + \sca{\nabla_1 Q(\theta', \theta') ; \theta-\theta'} + \frac 12 (\theta-\theta')^T \ub{\nabla_1 ^2 Q(\theta'|\theta')}_{\geq -L} (\theta-\theta')
        \\ & \geq Q(\theta'|\theta') + \sca{\nabla_1 Q(\theta', \theta') ; \theta-\theta'} - \frac L2 \norm{\theta-\theta'}^2
        \intertext{We conclude using the inequality : $2\sca{b,a} - \norm a^2 = \norm b^2 - \norm{a-b}^2$ with $a = \sqrt L(\theta-\theta')$ and $b = \frac 1{\sqrt L} \nabla Q(\theta'|\theta')$ }
         Q(\theta|\theta')  & \geq Q(\theta'|\theta') + \frac 1{2L} \norm{\nabla_1 Q(\theta'|\theta')}^2 - \frac L2\norm{\frac 1L \nabla_1 Q(\theta'|\theta') + \theta - \theta'}^2
    \end{align*}

    Then with $\frac 1\gamma \geq L$ : \fbox{$Q(\theta|\theta')  \geq Q(\theta'|\theta') + \frac \gamma2 \norm{\nabla_1 Q(\theta'|\theta')}^2 - \frac 1{2\gamma}\norm{\frac 1L \nabla_1 Q(\theta'|\theta') + \theta - \theta'}^2$}
    
    \end{dem}

    We place ourselves at iteration $k$ of a penalized EM with the current value $\theta' = \theta_k\times \setr^p$.
    //By the above lemma, we have for any $\gamma \in ]0,1/L]$ and  $\theta \in\setr^p$, 
    $$
        Q(\theta|\theta_k) -\pen(\theta) 
            \geq  Q(\theta_k|\theta_k) 
                   + \frac \gamma2 \norm{\nabla_1 Q(\theta_k|\theta_k)}^2 
                   - \frac 1{2\gamma}\norm{\gamma \nabla_1 Q(\theta_k|\theta_k) + \theta - \theta_k}^2 -\pen(\theta)
    $$

    Note that the equality case is achieve for $\theta = \theta_k$. 
    So if we take the $\theta$ that achieve the maxima at the right-hands side of the inequality :
    
    \begin{equation}\label{prox_arrival} \stack{
    \theta_{k+1} &= \argmax_{\theta\in\setr^p}\left\{ Q(\theta_k|\theta_k) + \frac \gamma2 \norm{\nabla_1 Q(\theta_k|\theta_k)}^2 - \tcr{\frac 1{2\gamma}\norm{\gamma \nabla_1 Q(\theta_k|\theta_k) + \theta - \theta_k}^2 -\pen(\theta)} \right\} 
        \\ &= \argmin_{\theta\in\setr^p} \left\{ \tcr{\frac 1{2\gamma}\norm{\gamma \nabla_1 Q(\theta_k|\theta_k) + \theta - \theta_k}^2 -\pen(\theta) }\right\}
        \\ & = \tcr{prox_{\gamma, \pen}\{\gamma \nabla_1 Q(\theta_k|\theta_k) + \theta_k \}  }  }
    \end{equation}
    
    So for any $\theta_{k+1}$ maximizing the right-hand side of the inequality, it holds the inequality, for any $\theta\in\setr^p$
    \begin{align*}
        Q(\theta_{k+1}|\theta_k) -\pen(\theta_{k+1}) 
        \geq & Q(\theta_k|\theta_k) 
                   + \frac \gamma2 \norm{\nabla_1 Q(\theta_k|\theta_k)}^2 
                   \\ &\qquad - \frac 1{2\gamma}\norm{\gamma \nabla_1 Q(\theta_k|\theta_k) + \theta_{k+1} - \theta_k}^2 -\pen(\theta_{k+1})
   \\   \geq & Q(\theta_k|\theta_k) 
                   + \frac \gamma2 \norm{\nabla_1 Q(\theta_k|\theta_k)}^2 
                   \\ &\qquad - \frac 1{2\gamma}\norm{\gamma \nabla_1 Q(\theta_k|\theta_k) + \theta - \theta_k}^2 -\pen(\theta)
    \end{align*}

    In particular if we take $\theta = \theta_k$ for the far right-hand side of the inequality, we can conclude : 

    \begin{center}
        \shabox{$Q(\theta_{k+1}|\theta_k) -\pen(\theta_{k+1})  \geq Q(\theta_k|\theta_k) -\pen(\theta_k) $}
    \end{center}
    
\end{dem}

\begin{rem}[on lipshitz hypo]

\end{rem}












\section{Model inspirer de \cite{he_simultaneous_2015}}\label{modelsection}

Rappel du modèle

\begin{myBoxedEnv}[final modeling]{Joint Model}{\bccrayon}{red}
    
    For any genotype $\foranygeno$ and observation $\foranyobs$
    
    \begin{equation}\label{jointmodel}
        \stackeq[lll]{ 
            \hazard_\geno (T_\geno|\mc M(T_\geno;\vphi_\geno), U_\geno) 
                = \hazard_0(T_\geno) \exp(\beta ^T U_\geno + \alpha \nonlinearfct(T_\geno;\vphi_\geno) )
        \\  Y_\genoobs = \nonlinearfct(t_\obs ; \vphi_\geno )  + \epsilon_\genoobs 
        \\  \vphi_\geno \sim \mc N(\mu, \Omega) \quad  ;  \quad  \epsilon_\genoobs \sim \mc N(0,\sigma^2) }
    \end{equation}
    
\end{myBoxedEnv}

\afaire{Pourquoi prendre $m(T_\geno; \vphi_\geno)$ alors que c'est une data observer (il me semble) ? \\ Si on a $n_g$ observation pour le génotype $g$ on pourrait remplacer cela par $Y_{\geno,n_\geno}$ ?}

Modèle par curiosité, inspirer de \cite{he_simultaneous_2015} :

For the longitudinal outcome, we consider the following non-linear mixed effects model : 

$$\stackeq[lll]{ Y_\genoobs =& \nonlinearfct(t_\obs ; \vphi_\geno )  + \epsilon_\genoobs
                        \\   \vphi_\geno = &\beta_1 ^T X_{1\geno} + b_\geno^T \Gamma_1 Z_{1\geno} }
                        \\ & \epsilon_\genoobs \sim \mc N(0,\sigma^2) ~;~ b_\geno \sim \mc N(0, I_q)$$

where $\beta_1 \in \setr^p$ the first regression parameter, $\Gamma_1$ is a $q\times q$ lower triangular matrix.

For the survival outcome, we consider a frailty model, defined as follows : 

$$\hazard_\geno (T_\geno) = \hazard_0(T_\geno) \exp(\beta_2 ^T x_{2\geno} + b_\geno^T \Gamma_2 z_{2\geno})$$

where $\hazard_0$ is the baseline hazard function, $\beta_2 \in \setr^p$ the second regression parameter, $\Gamma_2$ is a $q\times q$ lower triangular matrix.
                        
The combine observation will be $\mc O_\geno = (Y_\geno, X_{1\geno}, Z_{1\geno}, T_\geno, X_{2\geno}, Y_{2\geno})$

\afaire{Je sais pas si ce que je viens décrire est possible ni si cela à un sens en modélisation, en tout computationnellement parlant il y a un avantage indégniable :  il n'y aura pas d'intégrale à calculer dans la likelihood !}











\end{myText}















    \newpage
    \printbibliography

    \newpage
    
    \begin{myAppendix}

        \subsection{SAEM with gradient descent step}
            \subsubsection{Standard SAEM in exponential context}

            Assuming we got observation written $\varobs$ with unobserved latent variable $\varlat$ in a model characterize by the parameter $\theta$. We denote by $\Lcomp$ the complete likelihood of $\varobs$. We place ourselves in the framework of exponential families and write the complete likelihood as follows: \fbox{$\Lcomp[,\varobs](\theta) = \sca{\Phi(\theta) ; S(\varobs,\varlat)}- \psi(\theta)$}. Then, we can write the A-step as : 
            $$S_{k+1} = (1-u_k) S_k + u_k S(\varobs, \varlat_{k+1})$$
            
            The SAEM in the context of the exponential family can be written as follows: 
            
            \begin{myAlgorithm}[13cm]
                \caption{SAEM for exponential family}
                \Require{ Number of iterations $K\geq 1$ ; $(u_k)_{k\geq 1}$ a sequence of step-size}
                \Initialize Starting point $\theta_0 \in \setr^d$ ; $S_0 = 0$ ; $H_0 = 0$
                
                \For{$k=1$ \KwTo $K$}{
                    \myBullet[red]\bf{Step S :} \\ 
                    \quad simulate $\varlat[^{(k)}]$ according to the conditional distribution $f_{\varlat}(\varlat|\varobs, \theta_k)$
                     \\
                     \myBullet[red] \bf{Step A :} \\
                     \quad update $S_{k+1}$ \iffalse and $H_{k+1}$\fi according to a stochastic approximation :
                        $$S_{k+1} = (1-u_k) S_k + u_k S(\varobs, \varlat_{k+1}) $$
                        %$$H_{k+1} = (1-u_k) H_k + u_k H(\varobs, \varlat_{k+1}) $$
                     \\
                     \myBullet[red] \bf{Step M :} \\
                     \quad maximization : $\theta_{k+1} = \argmax_{\theta\in\Theta} \left\{\iffalse H_{k+1} +\fi \sca{ \Phi(\theta) ; S_{k+1}} - \psi(\theta) \right\}$
                }
                 \Return{$\hat \theta = \theta_K$}
            \end{myAlgorithm}
            
            Where $(u_k)_{k\geq 1}$ is a step-size sequence such that $\forall k \in \setn, u_k \in [0,1], \som k1\infty u_k= \infty$ and $\som k1\infty u_k^2< \infty$.
            
            

        \subsubsection{Adding gradient descent in maximization step}

We assume that we can have, on the one hand, real parameters and, on the other hand, high dimensional parameters in $\theta$. These last ones require a particular attention during the maximization that we wish to obtain by gradient descent. We denote by $\nu$ the parameters whose maximization is explicit and by $\beta$ for the others : $\theta = (\nu, \beta)\in V \times B = \Theta \subset\setr^{d+p}$.


To handle the high dimension we want to optimize the penalized likelihood as follows: 
    $$\hat \theta = (\hat\nu, \hat\beta) = \argmax_{\nu, \beta \in V \times B}\Lmarg[,\varobs] (\nu,\beta) - pen(\beta)$$

Where $pen$ might be the LASSO penalty $pen(\beta) = \lambda \norm\beta_1$ with $\lambda$ a regularization parameter to be determined. We propose to compute the maximization step of the SAEM by performing a \textbf{two-step maximization}. We first calculate the explicit maxima over $\nu$ with $\beta$ fixed equal to $\beta_k$ :

    $$\nu_{k+1} = \argmax_{\nu\in V}\left\{\sca{ \Phi(\nu,\beta_k) ; S_{k+1}} - \psi(\nu, \beta_k) \right\}$$ 

Then, using a specific method, we compute the maximization over the high-dimensional parameter $\beta$ with $\nu$ being fixed equal to the updated value $\nu_{k+1}$ as : 
    $$\beta_{k+1} = \argmax_{\beta\in B}\ub{\big\{\sca{ \Phi(\nu_{k+1},\beta) ; S_{k+1}} - \psi(\nu_{k+1}, \beta)}_{=Q_{k+1}(\beta)} - pen(\beta)\big\}$$

This last step is achieve with a proximal gradient descent as follow : 

\begin{myAlgorithm}[12cm]
    \caption{Proximal Normalized Gradient Descent}\label{PNGD}
    \Require{Number of iterations $K_{grad}\geq 1$, a function to optimize $Q$}
    \Initialize $\beta_0 \in \setr^p$ a starting point
    
    \For{$k=1$ \KwTo $K_{grad}$}{
        $\omega_k \leftarrow \beta_{k-1} - \gamma_k \frac{\nabla Q(\beta_{k-1})}{\norm{\nabla Q(\beta_{k-1})}_2}$
        
        $\beta_k \leftarrow prox_{\gamma_kpen}(\omega_k)$
    }
     \Return{$\beta_{K_{grad}}$}
\end{myAlgorithm}

The SAEM become the following Stochastic Approximation Expectation Gradient Descent Maximization 


\begin{myAlgorithm}[13cm]
    \caption{SAEGDM }\label{SARGDM}
    \Require{Number of iterations $K\geq 1$ ; $(u_k)_{k\geq 1}$ a sequence of step-size}
    \Initialize Starting point $\theta_0 \in \setr^d$ ; $S_0 = 0$ ; $H_0 = 0$
    
    \For{$k=1$ \KwTo $K$}{
        \myBullet[red]\bf{Step S :} \\ 
        \quad simulate $\varlat[^{(k)}]$ according to the conditional distribution $f_{\varlat}(\varlat|\varobs, \theta_k)$
         \\
         \myBullet[red] \bf{Step A :} \\
         \quad update $S_{k+1}$ and $H_{k+1}$ according to a stochastic approximation :
            $$S_{k+1} = (1-u_k) S_k + u_k S(\varobs, \varlat_{k+1}) $$
            %$$H_{k+1} = (1-u_k) H_k + u_k H(\varobs, \varlat_{k+1}) $$
         \\
         \myBullet[red] \bf{Step M :} \\
         \quad \myBullet[blue] maximization : $\nu_{k+1} = \argmax_{\nu\in V}\left\{\sca{ \Phi(\nu,\beta_k) ; S_{k+1}} - \psi(\nu, \beta_k) \right\}$
         
         \quad \myBullet[blue] gradient descent on $Q_{k+1}$: $$\beta_{k+1} = \argmax_{\beta\in B}\big\{\ub{\sca{ \Phi(\nu_{k+1},\beta) ; S_{k+1}} - \psi(\nu_{k+1}, \beta)}_{=Q_{k+1}(\beta)}- pen(\beta)\big\} $$
    }
     \Return{$\hat \theta = (\hat\nu,\hat\beta) = (\nu_K,\beta_K)$}
\end{myAlgorithm}


\end{myAppendix}










\end{document}