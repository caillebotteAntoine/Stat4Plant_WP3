\documentclass[a4paper]{article}
\usepackage[boxedEnv, numbering, titleFormat, english]{myBelovedPackage}

\margin{3cm}{4cm}
\setlhead{Antoine Caillebotte}

\loadbiblatex[citestyle = authoryear]{references.bib}



% Vraisemblance
\newcommand{\Lmarg}[1][]{\mc L_{marg#1}}
\newcommand{\logmarg}[1][]{\ell_{marg#1}}

\newcommand{\Lcomp}[1][]{\mc L_{comp#1}}
\newcommand{\logcomp}[1][]{\ell_{comp#1}}

\newcommand{\varobs}{\mc O}
\newcommand{\varlat}[1][]{Z#1}
\newcommand{\pen}{\mathrm{pen}}
\DeclareMathOperator*{\prox}{Prox}


\newcommand{\hazard }{h}
\newcommand{\regularization }{\lambda}
\newcommand{\nonlinearfct}{m}


\newcommand{\forany}[3][1]{#1\leq #2 \leq #3}
%Indice
\newcommand{\geno}{g}
\newcommand{\maxgeno}{G}
\newcommand{\foranygeno}{\forany{\geno}{\maxgeno}}

\newcommand{\lines}{\ell}
\newcommand{\maxlines}{L}
\newcommand{\foranylines}{\forany{\lines}{\maxlines}}

\newcommand{\genolines}{ {\geno,\lines} }
\newcommand{\meltgenolines}{ {\geno\lines} }

\newcommand{\obs}{j}
\newcommand{\maxobs}{J}
\newcommand{\foranyobs}{\forany{\obs}{\maxobs}}

\newcommand{\genolinesobs}{ {\geno, \lines, \obs} }
\newcommand{\genoobs}{ {\geno, \obs} }




% --- lazy math command --- %
\newcommand{\psinorm}[2]{\frac 12 \log(2\pi#1^2) + \frac{#2^2}{2#1^2}}
\newcommand{\psibar}[1]{\psinorm{\bar #1}{\sigma_{#1}}}



% Petite commande pour faire des model exp = h + <S, Phi> + psi parce que je suis feignant
\newcounter{Sc}
\setcounter{Sc}{0}
\newcommand{\Scount}{{\stepcounter{Sc}\theSc}}

\newcounter{Phic}
\setcounter{Phic}{0}
\newcommand{\Phicount}{{\stepcounter{Phic}\thePhic}}

\newcounter{hc}
\setcounter{hc}{0}
\newcommand{\hcount}{{\stepcounter{hc}\thehc}}

\newcounter{psic}
\setcounter{psic}{0}
\newcommand{\psicount}{{\stepcounter{psic}\thepsic}}
    

\renewcommand{\pen}{g}
\renewcommand{\logmarg}[1][]{\ell}

\begin{document} 
\begin{myText}

Assuming that the A step of the SAEM in the general context is written : 

$$Q_{k+1} (\theta) = (1-u_k)Q_k (\theta) + u_k \logcomp(\theta; \varlat[^{(k)}];\varobs)$$

where $\logcomp = \log\logcomp$, $\varlat$ is some latent variable and $\varobs$ the observation.


\begin{prop}
    $$Q_{k+1} (\theta) = \left(\pro l1{k+1} (1-u_{k+1})Q_l (\theta)\right) + \som l1{k+1} \left(\pro hl{k+1} (1-u_h)\right) \times u_l \logcomp(\theta; \varlat[^{(l)}];\varobs)$$
\end{prop}

\begin{dem}

    Let's say that the recurrence proposition is the following
    $$\mc P_k : "Q_{k+1} (\theta) = \left(\pro l1{k+1} (1-u_l)Q_l (\theta)\right) + \som {l+1}1{k} \left(\pro hl{k+1} (1-u_h)\right) \times u_l \logcomp(\theta; \varlat[^{(l)}];\varobs)"$$ 

    For $k=0$, we have obviously that : $Q_1 = (1-u_1)Q_0(\theta) + u_1 \logcomp(\theta; \varlat[^{(l)}];\varobs) $ so $\mc P_0$ true !

    Let be a certain $k\geq 0$, suppose that $\mc P_k$ is true, we have : 

    \begin{align*}
        Q_{k+2} 
        =& (1-u_{k+2}) Q_{k+1} + u_{k+2} \logcomp(\theta; \varlat[^{(k+2)}];\varobs)
        \\
        =& (1-u_{k+2}) \left(\pro l1{k+1} (1-u_l)Q_l (\theta)\right) 
         \\ &+  (1-u_{k+2})\som {l+1}1{k} \left(\pro hl{k+1} (1-u_h)\right) \times u_l \logcomp(\theta; \varlat[^{(l)}];\varobs) 
         \\ &+ u_{k+2}\logcomp(\theta; \varlat[^{(k+2)}];\varobs)  
        \\
        =& \left(\pro l1{k+2} (1-u_l)Q_l (\theta)\right) 
         + \som {l+1}1{k+1} \left(\pro hl{k+2} (1-u_h)\right) \times u_l \logcomp(\theta; \varlat[^{(l)}];\varobs)
    \end{align*}
\end{dem}

Compared to the number of iterations $K$ of the SAEM the computation of $\logcomp$ takes a constant time, in $O(n)$ with $n$ denoting the number observation. Computing $Q_k$ takes about one factorial time $O(kn)$. To compare, the calculation of step S following a one step MCMC procedure, if we denote by $n$ the number of observation is compute in $O(n^2)$

\begin{dem}

\end{dem}






\end{myText}
\end{document}